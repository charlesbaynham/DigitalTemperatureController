\hypertarget{index_Introduction}{}\section{Introduction}\label{index_Introduction}
This Arduino library manages the toggling of arbitary pins at controllable rates. For example, it can be used to flash an L\+ED at different speeds according to the state of your device.

It works using interrupts and is designed to be lightweight when running, so will not interfere with existing code.

This library requires exclusive usage of T\+I\+M\+E\+R1 so is incompatible with libraries that also use this timer.\hypertarget{index_Usage}{}\section{Usage}\label{index_Usage}
Include the class by adding {\ttfamily \#include $<$\hyperlink{pin_toggler_8h}{pin\+Toggler.\+h}$>$} to your sketch.

Init the class by passing it an array of pins that will be toggled. E.\+g. \begin{DoxyVerb}int pins[3] = {13, A4, A5};
pinToggler<3>::init(pins);
\end{DoxyVerb}


The template parameter ({\ttfamily $<$3$>$} above) defines the total number of pins being controlled.

These pins will start {\ttfamily L\+OW}, not toggling.

To start the toggling, set their speed to {\ttfamily O\+FF}, {\ttfamily S\+L\+OW}, {\ttfamily M\+E\+D\+I\+UM}, {\ttfamily F\+A\+ST} or {\ttfamily M\+AX}. E.\+g. \begin{DoxyVerb}pinToggler<3>::setFlashRate(0, SLOW);
\end{DoxyVerb}


N.\+B. The template parameter (here {\ttfamily $<$3$>$}) must match the one used in \hyperlink{classpin_toggler_aa58211817601d0e2dd413eb63e41f773}{pin\+Toggler\+::init()} or this will throw an error. Also the L\+ED parameter in \hyperlink{classpin_toggler_a42ae0ac4f841dae592195f26bffb109e}{pin\+Toggler\+::set\+Flash\+Rate} refers to the pins passed to \hyperlink{classpin_toggler_aa58211817601d0e2dd413eb63e41f773}{pin\+Toggler\+::init()}, zero-\/indexed in the order that they appeared in in the array.

The speeds refer to fractions of the max speed, defined by \hyperlink{pin_toggler_8h_a1097b75ec2a0f9ad2e6d4f6c4500d224}{F\+L\+A\+S\+H\+\_\+\+F\+R\+E\+Q\+\_\+\+HZ}.

\begin{DoxyWarning}{Warning}
Note that all the function calls here are static members of the class. Although a class object is created, this happens internally. For memory management purposes, be aware that this class allocates {\ttfamily 3 $\ast$ num\+Pins} bytes on the heap. Thus, to avoid memory framentation, \hyperlink{classpin_toggler_aa58211817601d0e2dd413eb63e41f773}{pin\+Toggler\+::init()} should be called as early in your code as possible.
\end{DoxyWarning}
\begin{DoxyCopyright}{Copyright}
Charles Baynham 2016 
\end{DoxyCopyright}
